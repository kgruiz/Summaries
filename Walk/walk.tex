\documentclass{article}
\usepackage[a4paper,margin=1in]{geometry}
\usepackage{listings}
\usepackage{xcolor}
\usepackage{hyperref}
\usepackage{titlesec}
\usepackage{tocloft}

% Define JavaScript and Rust languages for listings
\lstdefinelanguage{JavaScript}{%
    keywords={const, let, var, function, if, else, return, for, while},
    sensitive=true,
    morecomment=[l]{//},
    morecomment=[s]{/*}{*/},
    morestring=[b]",
    morestring=[b]'
}

\lstdefinelanguage{Rust}{%
    keywords={fn, let, mut, pub, impl, use, mod, crate, for, while, if, else, loop},
    sensitive=true,
    morecomment=[l]{//},
    morecomment=[s]{/*}{*/},
    morestring=[b]",
    morestring=[b]'
}

% Enhanced Code listing style
\lstset{%
    basicstyle=\ttfamily\small,
    keywordstyle=\color{blue}\bfseries,
    stringstyle=\color{red},
    commentstyle=\color{gray}\itshape,
    showstringspaces=false,
    breaklines=true,
    frame=lines,
    framesep=5pt,
    rulecolor=\color{black},
    backgroundcolor=\color{lightgray!10},
    numbers=left,
    numberstyle=\tiny\color{gray},
    stepnumber=1
}

% Section formatting
\titleformat{\section}{\bfseries\Large\sffamily\color{blue}}{}{0em}{}
\titleformat{\subsection}{\bfseries\large\sffamily\color{teal}}{}{0em}{}
\titleformat{\subsubsection}{\bfseries\normalsize\sffamily\color{purple}}{}{0em}{}

\title{Walking a File Tree in Popular Languages}
\author{}
\date{}

\begin{document}

\maketitle

\section*{Table of Contents}
\begin{enumerate}
    \item Python
    \item C++
    \item Zsh
    \item Bash
    \item JavaScript (Node.js)
    \item Rust
    \item Zig
    \item Go
    \item C\#
    \item Java
    \item Ruby
    \item PHP
\end{enumerate}

\newpage

\section{Python}
\subsection{Using \texttt{os.walk}}
\begin{lstlisting}[language=Python]
import os

for root, dirs, files in os.walk("/path/to/directory"):
    print(f"Directory: {root}")
    print(f"Subdirectories: {dirs}")
    print(f"Files: {files}")
\end{lstlisting}

\subsection{Using \texttt{pathlib}}
\begin{lstlisting}[language=Python]
from pathlib import Path

for path in Path("/path/to/directory").rglob("*"):
    print(path)
\end{lstlisting}

\section{C++}
\begin{lstlisting}[language=C++]
#include <iostream>
#include <filesystem>
namespace fs = std::filesystem;

int main() {%
    for (const auto& entry : fs::recursive_directory_iterator("/path/to/directory")) {%
        std::cout << entry.path() << std::endl;
    }
    return 0;
}
\end{lstlisting}

\section{Zsh}
\subsection{Recursive Globbing with \texttt{**}}
\begin{lstlisting}
for file in /path/to/directory/**/*; do
  echo $file
done
\end{lstlisting}

\subsection{Using \texttt{find}}
\begin{lstlisting}
for file in $(find /path/to/directory -type f); do
  echo $file
done
\end{lstlisting}

\subsection{Using Built-in Functions}
\begin{lstlisting}
walk_tree() {%
  local dir=$1
  for file in $dir/**/*; do
    [[ -f $file ]] && echo "File: $file"
    [[ -d $file ]] && echo "Directory: $file"
  done
}

walk_tree /path/to/directory
\end{lstlisting}

\section{Bash}
\begin{lstlisting}
find /path/to/directory -type f -print
\end{lstlisting}

\section{JavaScript (Node.js)}
\begin{lstlisting}[language=JavaScript]
const fs = require('fs');
const path = require('path');

function walk(dir) {%
    fs.readdirSync(dir).forEach(file => {%
        const fullPath = path.join(dir, file);
        if (fs.statSync(fullPath).isDirectory()) {%
            walk(fullPath);
        } else {%
            console.log(fullPath);
        }
    });
}

walk('/path/to/directory');
\end{lstlisting}

\section{Rust}
\begin{lstlisting}[language=Rust]
use walkdir::WalkDir;

fn main() {%
    for entry in WalkDir::new("/path/to/directory") {%
        let entry = entry.unwrap();
        println!("{}", entry.path().display());
    }
}
\end{lstlisting}

\section{Zig}
\begin{lstlisting}
const std = @import("std");

pub fn main() !void {%
    const cwd = try std.fs.cwd();
    var it = try cwd.walk("/path/to/directory", .{});
    while (try it.next()) |entry| {%
        if (entry.kind == .File) {%
            std.debug.print("File: {s}\n", .{entry.path});
        } else if (entry.kind == .Directory) {%
            std.debug.print("Directory: {s}\n", .{entry.path});
        }
    }
}
\end{lstlisting}

\section{Go}
\begin{lstlisting}[language=Go]
package main

import (
	"fmt"
	"os"
	"path/filepath"
)

func main() {%
	root := "/path/to/directory"
	err := filepath.Walk(root, func(path string, info os.FileInfo, err error) error {%
		if err != nil {%
			return err
		}
		fmt.Println(path)
		return nil
	})
	if err != nil {%
		fmt.Println("Error:", err)
	}
}
\end{lstlisting}

\section{C\#}
\subsection{Using \texttt{Directory.GetFiles}}
\begin{lstlisting}[language=C]
using System;
using System.IO;

class Program {%
    static void Main() {%
        foreach (string file in Directory.GetFiles("/path/to/directory", "*", SearchOption.AllDirectories)) {%
            Console.WriteLine(file);
        }
    }
}
\end{lstlisting}

\section{Java}
\begin{lstlisting}[language=Java]
import java.nio.file.*;
import java.io.IOException;

public class FileTreeWalk {%
    public static void main(String[] args) throws IOException {%
        Path path = Paths.get("/path/to/directory");
        Files.walk(path).forEach(System.out::println);
    }
}
\end{lstlisting}

\section{Ruby}
\subsection{Using \texttt{Dir.glob}}
\begin{lstlisting}[language=Ruby]
Dir.glob("/path/to/directory/**/*").each do |file|
  puts file
end
\end{lstlisting}

\subsection{Using \texttt{Find}}
\begin{lstlisting}[language=Ruby]
require 'find'

Find.find('/path/to/directory') do |path|
  puts path
end
\end{lstlisting}

\section{PHP}
\begin{lstlisting}[language=PHP]
<?php
$iterator = new RecursiveIteratorIterator(new RecursiveDirectoryIterator('/path/to/directory'));
foreach ($iterator as $file) {%
    echo $file . PHP_EOL;
}
?>
\end{lstlisting}

\end{document}

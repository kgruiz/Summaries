\documentclass[12pt,a4paper]{article}

%--------------------------
%--- PACKAGES & SETTINGS---
%--------------------------

\usepackage[margin=1in]{geometry}     % 1-inch margins
\usepackage[T1]{fontenc}             % Better font encoding
\usepackage{lmodern}                 % Modern Latin font
\usepackage{microtype}               % Improved text spacing
\usepackage{xcolor}                  % Colors

% Basic math and images
\usepackage{amsmath,amssymb}
\usepackage{graphicx}

% Hyperlinks
\usepackage{hyperref}
\hypersetup{%
  colorlinks = true,
  urlcolor   = teal,
  linkcolor  = teal,
  citecolor  = teal
}

% Minted for code, ensure to compile with --shell-escape
\usepackage{minted}

% Define custom colors for minted
\definecolor{MintedLightGray}{HTML}{F7F7F7}
\setminted{%
  style=friendly,
  fontsize=\small,
  breaklines=true,
  breakanywhere=true,
  frame=lines,
  framesep=0.7em,
  baselinestretch=1.1,
  bgcolor=MintedLightGray
}

% A single accent color for headings
\definecolor{accent}{HTML}{1F618D}

% Title and section formatting
\usepackage{titlesec}
\titleformat{\section}{\Large\bfseries\color{accent}}{\thesection}{1em}{}
\titleformat{\subsection}{\large\bfseries\color{accent}}{\thesubsection}{1em}{}

\renewcommand{\familydefault}{\sfdefault} % Sans-serif for a clean look
\setlength{\parskip}{0.8em}              % Extra spacing between paragraphs
\setlength{\parindent}{0pt}             % No indentation

% Simplified table style
\usepackage{array}

% Title info
\title{\Huge A Simple Guide to Safari Extensions}
\date{}   % No date
\author{} % No author name

%--------------------------
%--- DOCUMENT BEGIN -------
%--------------------------

\begin{document}

\maketitle

\section{What Are Safari Extensions?}
A Safari extension adds new features to Safari or modifies its behavior. It uses HTML, CSS, JavaScript, and JSON at its core. There are two main types:

\subsection{Safari Web Extensions}
These rely on the WebExtensions API, also used by Chrome, Firefox, and Edge, making them generally cross-browser.

\subsection{Safari App Extensions}
These live inside a macOS app and integrate deeply with Safari and macOS. Ideal if you need features that interact with the system at a lower level.

\section{Common Languages and Tools}
While Safari extensions are primarily written in HTML, CSS, JavaScript, and JSON, other languages can provide additional power:

\begin{itemize}
  \item \textbf{TypeScript:} Adds type safety to JavaScript.
  \item \textbf{WebAssembly:} Run C/C++/Rust code in the browser with near-native speed.
  \item \textbf{Swift:} Especially for Safari App Extensions on macOS/iOS.
  \item \textbf{Python:} Handy for build scripts, automation, or backend tasks.
  \item \textbf{C++:} Can be compiled to WebAssembly for computationally intensive tasks.
\end{itemize}

\subsection{Using Just Python or C++?}
You can’t build the entire extension in these languages alone. Browsers expect HTML, CSS, JavaScript, and JSON. Python or C++ is best used on the side: generating config files or doing heavy lifting compiled to WebAssembly.

\section{Pros and Cons of Key Languages}
\renewcommand{\arraystretch}{1.2}
\begin{center}
  \begin{tabular}{|m{3cm}|m{5cm}|m{5cm}|}
    \hline
    \textbf{Language} & \textbf{Pros}                      & \textbf{Cons}                  \\
    \hline
    HTML              & Universal layout                   & Static by itself               \\
    \hline
    CSS               & Powerful styling                   & Complex for large-scale design \\
    \hline
    JavaScript        & Widely supported, flexible         & No static typing by default    \\
    \hline
    TypeScript        & Type safety, large project scaling & Compiles to JavaScript         \\
    \hline
    WebAssembly       & Near-native speed                  & Requires JS integration        \\
    \hline
    Swift             & Deep Apple integration             & Apple-only                     \\
    \hline
    Python            & Great for scripting                & Not for the browser core       \\
    \hline
  \end{tabular}
\end{center}

\section{Accessing Websites and Data}
\subsection{Content Scripts}
Content scripts run in the context of a webpage’s DOM.

\begin{minted}{json}
{
  "manifest_version": 2,
  "name": "Content Script Example",
  "version": "1.0",
  "content_scripts": [
    {
      "matches": ["<all_urls>"],
      "js": ["content.js"]
    }
  ]
}
\end{minted}

\begin{minted}{javascript}
// content.js
console.log("Page title is:", document.title);
\end{minted}

\subsection{Browser APIs}
Use built-in APIs for data fetching or tab management:

\begin{minted}{javascript}
fetch("https://api.example.com/data")
  .then(r => r.json())
  .then(data => console.log(data))
  .catch(err => console.error("Fetch error:", err));
\end{minted}

\section{Integrating Python, Qt, and C++}
\subsection{Python for Scripting or Backend}
You might fetch data or manipulate files:

\begin{minted}{python}
import requests

res = requests.get("https://api.example.com/data")
if res.status_code == 200:
    print(res.json())
else:
    print("Error fetching data")
\end{minted}

\subsection{Qt for a Companion GUI}
If you need a desktop app to configure your extension:

\begin{minted}{python}
from PySide6.QtWidgets import (
    QApplication, QWidget, QVBoxLayout, QPushButton
)

app = QApplication([])
window = QWidget()
layout = QVBoxLayout()

button = QPushButton("Save Settings")
layout.addWidget(button)

window.setLayout(layout)
window.show()
app.exec()
\end{minted}

\subsection{C++ via WebAssembly for Performance}
Compile C++ to Wasm and call it from JavaScript.

\begin{minted}{cpp}
// mymodule.cpp
#include <emscripten/emscripten.h>

extern "C" {
  EMSCRIPTEN_KEEPALIVE
  int add(int a, int b) {
    return a + b;
  }
}
\end{minted}

Compile:
\begin{minted}{bash}
emcc mymodule.cpp -o mymodule.wasm -s EXPORTED_FUNCTIONS='["_add"]'
\end{minted}

Then use from JavaScript:
\begin{minted}{javascript}
fetch("mymodule.wasm")
  .then(r => r.arrayBuffer())
  .then(bytes => WebAssembly.instantiate(bytes))
  .then(result => {
    console.log(result.instance.exports.add(2, 3));
  });
\end{minted}

\section{Examples of Safari Web Extensions}
Below are additional examples that show how you might structure a simple Safari Web Extension. Each example uses standard web technologies, but you can adapt them to TypeScript, WebAssembly, or other languages as needed.

\subsection{Dark Mode Toggler}
This example toggles a dark mode style on any webpage. It includes a basic \texttt{manifest.json}, a background script, and a content script. The variable, function, and constant naming in the newly shown code follows a custom convention for clarity.

\begin{minted}{json}
{
  "manifest_version": 2,
  "name": "Dark Mode Toggler",
  "version": "1.0",
  "description": "Toggle dark mode on any page.",
  "permissions": [
    "activeTab"
  ],
  "browser_action": {
    "default_title": "Toggle Dark Mode",
    "default_icon": "icon.png"
  },
  "background": {
    "scripts": ["background.js"]
  },
  "content_scripts": [
    {
      "matches": ["<all_urls>"],
      "js": ["content.js"]
    }
  ]
}
\end{minted}

\begin{minted}{javascript}
// background.js
function ToggleDarkMode() {
  chrome.tabs.executeScript({
    file: "content.js"
  });
}

chrome.browserAction.onClicked.addListener(ToggleDarkMode);
\end{minted}

\begin{minted}{javascript}
// content.js
const DARK_MODE_CLASS = "dark-mode-active";

function TogglePageDarkMode() {
  let bodyElement = document.body;
  if (bodyElement.classList.contains(DARK_MODE_CLASS)) {
    bodyElement.classList.remove(DARK_MODE_CLASS);
  } else {
    bodyElement.classList.add(DARK_MODE_CLASS);
  }
}

TogglePageDarkMode();
\end{minted}

\begin{minted}{css}
/* content.css: example stylesheet if needed */
.dark-mode-active {
  filter: invert(1) hue-rotate(180deg);
}
\end{minted}

You can bundle the optional \texttt{content.css} if your extension needs more styling control. When the user clicks the extension's icon, it injects \texttt{content.js} which toggles a dark mode filter.

\subsection{Link Shortener Extension}
This extension fetches a short link from an API and inserts it into the current page’s DOM. It’s useful for quickly sharing long links. The \texttt{manifest.json} is similar, so we’ll just focus on the content script.

\begin{minted}{javascript}
// content.js
const API_URL = "https://api.tinyurl.com/create";
let currentUrl = window.location.href;

function FetchShortLink(urlToShorten) {
  return fetch(API_URL, {
    method: "POST",
    headers: { "Content-Type": "application/json" },
    body: JSON.stringify({
      url: urlToShorten
    })
  })
    .then(response => response.json());
}

function InsertShortLinkIntoPage(shortLink) {
  let messageContainer = document.createElement("div");
  messageContainer.style.position = "fixed";
  messageContainer.style.top = "10px";
  messageContainer.style.right = "10px";
  messageContainer.style.padding = "10px";
  messageContainer.style.backgroundColor = "#1F618D";
  messageContainer.style.color = "#FFFFFF";
  messageContainer.style.zIndex = 999999;
  messageContainer.textContent = "Short link: " + shortLink;
  document.body.appendChild(messageContainer);
}

FetchShortLink(currentUrl)
  .then(data => {
    if (data.data && data.data.tiny_url) {
      InsertShortLinkIntoPage(data.data.tiny_url);
    } else {
      console.error("Shortening failed:", data);
    }
  })
  .catch(error => console.error("Request error:", error));
\end{minted}

When the content script runs, it posts the page’s URL to a link-shortening API, then displays the short link in a fixed-position overlay. You could add a browser action or context menu to trigger it only when requested.

\section{Conclusion}
Safari extensions rely on standard web technologies, but you can use other languages to streamline workflows or speed up certain tasks. Whether you automate files with Python or bring in C++ via WebAssembly, you can combine these tools to build powerful Safari extensions that go beyond simple front-end scripting. The examples above demonstrate how you can inject and manage scripts, toggle styling, or integrate external APIs, and you can expand on them to create truly unique experiences in Safari.

\end{document}

\documentclass[12pt,a4paper]{article}

%--------------------------
%--- PACKAGES & SETTINGS---
%--------------------------

\usepackage[margin=1in]{geometry}     % 1-inch margins
\usepackage[T1]{fontenc}             % Better font encoding
\usepackage{lmodern}                 % Modern Latin font
\usepackage{microtype}               % Improved text spacing
\usepackage{xcolor}                  % Colors

% Basic math and images
\usepackage{amsmath,amssymb}
\usepackage{graphicx}

% Hyperlinks
\usepackage{hyperref}
\hypersetup{%
  colorlinks = true,
  urlcolor   = teal,
  linkcolor  = teal,
  citecolor  = teal
}

% Minted for code, ensure to compile with --shell-escape
\usepackage{minted}

% Define custom colors for minted
\definecolor{MintedLightGray}{HTML}{F7F7F7}
\setminted{%
  style=friendly,
  fontsize=\small,
  breaklines=true,
  breakanywhere=true,
  frame=lines,
  framesep=0.7em,
  baselinestretch=1.1,
  bgcolor=MintedLightGray
}

% A single accent color for headings
\definecolor{accent}{HTML}{1F618D}

% Title and section formatting
\usepackage{titlesec}
\titleformat{\section}{\Large\bfseries\color{accent}}{\thesection}{1em}{}
\titleformat{\subsection}{\large\bfseries\color{accent}}{\thesubsection}{1em}{}

\renewcommand{\familydefault}{\sfdefault} % Sans-serif for a clean look
\setlength{\parskip}{0.8em}              % Extra spacing between paragraphs
\setlength{\parindent}{0pt}             % No indentation

% Simplified table style
\usepackage{array}

% Title info
\title{\Huge A Simple Guide to Safari Extensions}
\date{}   % No date
\author{} % No author name

%--------------------------
%--- DOCUMENT BEGIN -------
%--------------------------

\begin{document}

\maketitle

\section{What Are Safari Extensions?}
A Safari extension adds new features to Safari or modifies its behavior. It uses HTML, CSS, JavaScript, and JSON at its core. There are two main types:

\subsection{Safari Web Extensions}
These rely on the WebExtensions API, also used by Chrome, Firefox, and Edge, making them generally cross-browser.

\subsection{Safari App Extensions}
These live inside a macOS app and integrate deeply with Safari and macOS. Ideal if you need features that interact with the system at a lower level.

\section{Common Languages and Tools}
While Safari extensions are primarily written in HTML, CSS, JavaScript, and JSON, other languages can provide additional power:

\begin{itemize}
    \item \textbf{TypeScript:} Adds type safety to JavaScript.
    \item \textbf{WebAssembly:} Run C/C++/Rust code in the browser with near-native speed.
    \item \textbf{Swift:} Especially for Safari App Extensions on macOS/iOS.
    \item \textbf{Python:} Handy for build scripts, automation, or backend tasks.
    \item \textbf{C++:} Can be compiled to WebAssembly for computationally intensive tasks.
\end{itemize}

\subsection{Using Just Python or C++?}
You can’t build the entire extension in these languages alone. Browsers expect HTML, CSS, JavaScript, and JSON. Python or C++ is best used on the side: generating config files or doing heavy lifting compiled to WebAssembly.

\section{Pros and Cons of Key Languages}
\renewcommand{\arraystretch}{1.2}
\begin{center}
    \begin{tabular}{|m{3cm}|m{5cm}|m{5cm}|}
        \hline
        \textbf{Language} & \textbf{Pros}                      & \textbf{Cons}                  \\
        \hline
        HTML              & Universal layout                   & Static by itself               \\
        \hline
        CSS               & Powerful styling                   & Complex for large-scale design \\
        \hline
        JavaScript        & Widely supported, flexible         & No static typing by default    \\
        \hline
        TypeScript        & Type safety, large project scaling & Compiles to JavaScript         \\
        \hline
        WebAssembly       & Near-native speed                  & Requires JS integration        \\
        \hline
        Swift             & Deep Apple integration             & Apple-only                     \\
        \hline
        Python            & Great for scripting                & Not for the browser core       \\
        \hline
    \end{tabular}
\end{center}

\section{Accessing Websites and Data}
\subsection{Content Scripts}
Content scripts run in the context of a webpage’s DOM.

\begin{minted}{json}
{%
  "manifest_version": 2,
  "name": "Content Script Example",
  "version": "1.0",
  "content_scripts": [
    {%
      "matches": ["<all_urls>"],
      "js": ["content.js"]
    }
  ]
}
\end{minted}

\begin{minted}{javascript}
// content.js
console.log("Page title is:", document.title);
\end{minted}

\subsection{Browser APIs}
Use built-in APIs for data fetching or tab management:

\begin{minted}{javascript}
fetch("https://api.example.com/data")
  .then(r => r.json())
  .then(data => console.log(data))
  .catch(err => console.error("Fetch error:", err));
\end{minted}

\section{Integrating Python, Qt, and C++}
\subsection{Python for Scripting or Backend}
You might fetch data or manipulate files:

\begin{minted}{python}
import requests

res = requests.get("https://api.example.com/data")
if res.status_code == 200:
    print(res.json())
else:
    print("Error fetching data")
\end{minted}

\subsection{Qt for a Companion GUI}
If you need a desktop app to configure your extension:

\begin{minted}{python}
from PySide6.QtWidgets import (
    QApplication, QWidget, QVBoxLayout, QPushButton
)

app = QApplication([])
window = QWidget()
layout = QVBoxLayout()

button = QPushButton("Save Settings")
layout.addWidget(button)

window.setLayout(layout)
window.show()
app.exec()
\end{minted}

\subsection{C++ via WebAssembly for Performance}
Compile C++ to Wasm and call it from JavaScript.

\begin{minted}{cpp}
// mymodule.cpp
#include <emscripten/emscripten.h>

extern "C" {%
  EMSCRIPTEN_KEEPALIVE
  int add(int a, int b) {%
    return a + b;
  }
}
\end{minted}

Compile:
\begin{minted}{bash}
emcc mymodule.cpp -o mymodule.wasm -s EXPORTED_FUNCTIONS='["_add"]'
\end{minted}

Then use from JavaScript:
\begin{minted}{javascript}
fetch("mymodule.wasm")
  .then(r => r.arrayBuffer())
  .then(bytes => WebAssembly.instantiate(bytes))
  .then(result => {%
    console.log(result.instance.exports.add(2, 3));
  });
\end{minted}

\section{Conclusion}
Safari extensions rely on standard web technologies, but you can use other languages to streamline workflows or speed up certain tasks. Whether you automate files with Python or bring in C++ via WebAssembly, you can combine these tools to build powerful Safari extensions that go beyond simple front-end scripting.

\end{document}
